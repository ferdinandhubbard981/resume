\documentclass{resume} % Use the custom resume.cls style

\usepackage[left=0.4 in,top=0.1in,right=0.4 in,bottom=0.1in]{geometry} % Document margins
\newcommand{\tab}[1]{\hspace{.2667\textwidth}\rlap{#1}} 
\newcommand{\itab}[1]{\hspace{0em}\rlap{#1}}
\name{Ferdinand Hubbard} % Your name
% You can merge both of these into a single line, if you do not have a website.
\address{+32 479 820 811 \\ Bristol, UK} 
\address{\href{mailto:ej21378@bristol.ac.uk}{ej21378@bristol.ac.uk} \\ \href{https://www.linkedin.com/in/ferdinand-hubbard/}{https://www.linkedin.com/in/ferdinand-hubbard/}} 

\begin{document}

%----------------------------------------------------------------------------------------
%	OBJECTIVE
%----------------------------------------------------------------------------------------


{Born in Belgium, I came to the UK to pursue a career in software engineering. To that end, I've developed my skills by working on challenging projects, and by being part of a software engineering team at FN Herstal.}


%----------------------------------------------------------------------------------------
%	EDUCATION SECTION
%----------------------------------------------------------------------------------------

\begin{rSection}{Education}

{\bf Master of Engineering (MEng) Computer Science} \hfill {September 2021 - June 2025} \\
\textit{University of Bristol} \hfill \textit{Bristol, UK} \\
Third year options: machine learning, types \& lambda calculus, artificial intelligence, computer graphics, applied data science

{\bf A levels} \hfill {September 2016 - June 2021} \\
\textit{Ampleforth College} \hfill \textit{Yorkshire, UK} \\
Further mathematics, mathematics, computer science, physics: A*, A*, A*, A

\end{rSection}

%----------------------------------------------------------------------------------------
% TECHNICAL STRENGTHS	
%----------------------------------------------------------------------------------------
\begin{rSection}{SKILLS}

{\bf Programming languages:} C++, C, python, golang, java \\
{\bf Web development languages:} HTML \& CSS, javascript, typescript \\
{\bf Tools:} kubernetes, docker, git, LaTeX \\
{\bf Languages:} bilingual in English and French

\end{rSection}

\begin{rSection}{EXPERIENCE}

\textbf{Software Engineering intern} \hfill June 2023 - August 2023\\
\textit{FN Herstal} \hfill \textit{Herstal, Belgium} \\
As a member of the core software engineering team, I took the lead in designing a system to generate and validate a static analysis configuration. Our goal was to automate adherence to the coding standard as much as possible, whilst identifying the rules that required human intervention for enforcement.

\textbf{Teaching assistant} \hfill September 2022 - June 2023\\
\textit{University of Bristol} \hfill \textit{Bristol, UK} \\
I supported the teaching of the first year computer science 'Imperative programming' and 'Mathematics B' units.
 
\end{rSection} 

%----------------------------------------------------------------------------------------
%	WORK EXPERIENCE SECTION
%----------------------------------------------------------------------------------------

\begin{rSection}{PROJECTS}
\vspace{-1.25em}

\item \textbf{recallai.app (Kubernetes, python, typescript):} \hfill May 2023 - \textit{current} \\
{My co-founder and I have developed a website that generates flashcards from lectures and videos. It leverages existing ML models to generate a transcript, and then condense that information into flashcards. The app is hosted in a bare-metal k8s instance, and was developed to be horizontally scalable.}

\item \textbf{Composite design tools (typescript, react):} \hfill September 2022 - May 2023 \\
{My team and I worked with Imperial University's material science department to develop a website that packaged compressive strength formulas into models. These models were then used to plot graphs such that meaningful conclusions could be drawn from them.}

\item \textbf{Pregame analysis for video game 'League of Legends' (Python, Google Colab, TensorFlow):} \hfill 2021 \\
{I collected and processed millions of pre-game data to train a neural network (generated using a Bayesian optimization algorithm) to predict the outcome of a game.}% \href{https://hiring-search.careerflow.ai/}{(Try it here)}}

\item \textbf{Maths Toolkit (C++, JavaScript):} \hfill 2021 \\
{I developed a tool aimed at A-level Further Maths students to aid their learning of the Simplex algorithm, matrices and solving polynomials numerically.}

\end{rSection}

%----------------------------------------------------------------------------------------
\begin{rSection}{Extra-Curricular Activities} 
\vspace{-1.25em}

\item{\textbf{Sports:}}
I am an active member of the tennis and kickboxing societies at the university of bristol, and also like running and climbing. \\

\vspace{-1.25em}

\item{\textbf{Miscellaneous:}}
I took part in a CTF organized by BAE Systems and the Bristol computer science society. \\

\end{rSection}


\end{document}
